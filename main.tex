\documentclass{article}

\usepackage[T1]{fontenc}
\usepackage[french]{babel}

\usepackage[letterpaper,top=2cm,bottom=2cm,left=3cm,right=3cm,marginparwidth=1.75cm]{geometry}

\usepackage{amsmath}
\usepackage{graphicx}
\usepackage[colorlinks=true, allcolors=blue]{hyperref}
\usepackage{eurosym}
\usepackage[group-separator={\ },output-decimal-marker={,}]{siunitx} 
\usepackage{amsfonts}
\usepackage{float}
\usepackage{pdflscape}
\usepackage{algorithm}
\usepackage{algorithmic}
\usepackage{array}
\usepackage{mdframed}
\usepackage{enumitem}
\usepackage{tcolorbox} % Pour les boîtes colorées
\usepackage{fontawesome5} % Pour les icônes
\usepackage{xcolor} % Pour les couleurs

\definecolor{lightgreen}{RGB}{230, 255, 230}

\newenvironment{conditions}
  {\par\vspace{\abovedisplayskip}\noindent\begin{tabular}{>{$}l<{$} @{${}={}$} l}}
  {\end{tabular}\par\vspace{\belowdisplayskip}}

\title{Impact de la fréquence de capitalisation sur le taux d'intérêt effectif et la valorisation des investissements}
\author{Gaëtan Bouget}

\begin{document}
\maketitle

\section{Problème}
Dans un contexte financier caractérisé par une diversification accrue des produits d’investissement, la variation des fréquences de capitalisation complique l’évaluation objective des rendements réels. Ainsi, deux actifs affichant un même taux nominal annuel peuvent générer des rendements distincts en raison des différentes fréquences de capitalisation, amplifiant ainsi l'effet des intérêts composés.

\section{Objectif}
L'objectif consiste à fournir une compréhension approfondie des mécanismes influençant le rendement réel des investissements et à établir une base solide pour la comparaison de diverses stratégies financières.

\section{Méthode}
L'approche adoptée pour atteindre l'objectif se décline en quatre étapes :
\begin{enumerate}
    \item calcul du capital final pour les actifs à intérêts simples ;
    \item prise en compte des intérêts composés dans l’analyse comparative ;
    \item démonstration de la supériorité des intérêts composés par développement binomial ;
    \item établissement de la formule du taux effectif annuel (TEA).
\end{enumerate}

Afin de faciliter les applications numériques avant de généraliser, l'analyse se concentre sur l'investissement d'un capital initial de \( C_0 = 100\,000\ \text{€} \), réparti sur quatre actifs décrits dans le tableau~\ref{tab:scenarios}.

\begin{table}[h!]
\centering
\begin{tabular}{|c|c|c|c|}
\hline
\textbf{Actif} & \textbf{Type d'intérêt} & \textbf{Fréquence de capitalisation} & \textbf{Taux périodique} \\
\hline
\(A_q^s\) & Intérêts simples & Quotidienne & \(r_\text{quotidien} = \frac{5}{365}\%\) \\
\hline
\(A_a^s\) & Intérêts simples & Annuelle & \(r_\text{annuel} = 5\%\) \\
\hline
\(A_q^c\) & Intérêts composés & Quotidienne & \(r_\text{quotidien} = \frac{5}{365}\%\) \\
\hline
\(A_a^c\) & Intérêts composés & Annuelle & \(r_\text{annuel} = 5\%\) \\
\hline
\end{tabular}
\caption{Scénarios d'investissement avec un capital initial \(C_0 = 100\,000\ \text{€}\).}
\label{tab:scenarios}
\end{table}

Chaque étape de l'analyse sera ensuite détaillée à travers une série de questions destinées à approfondir la compréhension du phénomène.

\subsection{Hypothèses de travail}
Les conditions suivantes sont posées :
\begin{itemize}
    \item Les intérêts sont calculés selon la fréquence de capitalisation définie pour chaque actif. Aucune capitalisation intermédiaire ni prorata temporis ne s'applique en cas de retrait avant la fin de la période de capitalisation.
    \item Une année est considérée comme comprenant 365 jours.
    \item Aucun frais ni taxe n'est appliqué.
    \item Aucun versement ou retrait supplémentaire n’est effectué durant la période d'investissement (seul le capital initial \( C_0 \) est investi).
    \item Le taux de rendement est constant dans le temps.
    \item L'inflation n'est pas prise en compte.
\end{itemize}

\subsection{Questions}
\subsubsection*{Étape 1 : intérêts simples}

\begin{tcolorbox}[
        colback=lightgreen, 
        colframe=lightgreen, 
        boxrule=0.5pt, 
        arc=0pt, 
        left=10pt, 
        right=10pt, 
        top=6pt, 
        bottom=6pt, 
        boxsep=2pt, 
        before upper={\faLightbulb\hspace{10pt}}
    ]
        Les intérêts simples sont calculés uniquement sur le capital initial, et le montant des intérêts gagnés chaque période est constant. La formule suivante reflète bien cette caractéristique, car le produit \( r \times t \) est simplement ajouté au capital initial :

        \[
        C^{\text{simple}}(t) = C_0 \left(1 + r \times t\right)
        \]
        
        où :
        \begin{itemize}
            \item \( C^{\text{simple}}(t) \) est le capital accumulé après une période \( t \).
            \item \( C_0 \) est le capital initial investi.
            \item \( r \) est le taux d'intérêt nominal (exprimé sous forme décimale, par exemple, 5\% devient 0,05).
            \item \( t \) est le temps écoulé, qui peut être exprimé en années ou en jours, selon la fréquence de capitalisation.
        \end{itemize}
    \end{tcolorbox}

\begin{enumerate}[label=\textbf{Q\arabic*.}]
    \item Soit un capital initial \( C_0 = 100\,000\ \text{€} \) investi, calculer le capital final pour l'actif \( A_q^s \) après une durée \( t \) égale à :
    \begin{enumerate}[label=(\alph*)]
        \item \( t = 1 \) jour
        \item \( t = 30 \) jours
        \item \( t = 1 \) an
        \item \( t = 30 \) ans
    \end{enumerate}

    \item Soit un capital initial \( C_0 = 100\,000\ \text{€} \) investi, calculer le capital final pour l'actif $A_a^s$ après une durée \( t \) égale à :
    \begin{enumerate}[label=(\alph*)]
        \item \( t = 1 \) jour
        \item \( t = 30 \) jours
        \item \( t = 1 \) an
        \item \( t = 30 \) ans
    \end{enumerate}
    
    \item Remplir le tableau :\\
    \begin{table}[h!]
        \centering
        \begin{tabular}{|c|c|c|c|c|}
        \hline
        \textbf{Actif} & \textbf{1 jour} & \textbf{30 jours} & \textbf{1 an} & \textbf{30 ans} \\
        \hline
        $A_q^s$ & & & & \\
        \hline
        $A_a^s$ & & & & \\
        \hline
        \end{tabular}
        \caption{Capital final pour différentes durées d'investissement avec intérêts simples et un capital initial de \( C_0 = 100\,000\ \text{€} \).}
        \label{tab:simple_interest_results}
    \end{table}

    \item Soit un capital initial \( C_0 = 100\,000\ \text{€} \) et un taux nominal annuel de 5\%, exprimer les fonctions permettant de calculer le capital final en fonction de la durée d'investissement \( t \) (exprimée en jours) pour chacun des deux actifs à intérêts simples :
    \begin{enumerate}[label=(\alph*)]
        \item \( C^s_q(t) \) pour l'actif à capitalisation quotidienne ;
        \item \( C^s_a(t) \) pour l'actif à capitalisation annuelle.
    \end{enumerate}

    \item Tracer les courbes \( C^s_q(t) \) et \( C^s_a(t) \) en représentant la durée d'investissement \( t \) (exprimée en années) en abscisse et le capital final (en €) en ordonnée, pour \( t \) variant de 0 à 40 ans. Commenter l'évolution du capital au fil du temps.

\end{enumerate}

\section{Solution}


\section{Prochaine étape}  
Développer un programme de simulation permettant de modéliser l'évolution du capital en fonction du montant investi, du taux de rendement, de la fréquence de capitalisation des intérêts et de la durée d'investissement. Le programme devra également calculer et afficher le taux d'intérêt effectif.  

\end{document}
