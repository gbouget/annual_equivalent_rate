\documentclass{article}

\usepackage[T1]{fontenc}
\usepackage[french]{babel}

\usepackage[letterpaper,top=2cm,bottom=2cm,left=3cm,right=3cm,marginparwidth=1.75cm]{geometry}

\usepackage{amsmath}
\usepackage{graphicx}
\usepackage[colorlinks=true, allcolors=blue]{hyperref}
\usepackage{eurosym}
\usepackage[group-separator={\ },output-decimal-marker={,}]{siunitx} 
\usepackage{amsfonts}
\usepackage{float}
\usepackage{pdflscape}
\usepackage{algorithm}
\usepackage{algorithmic}
\usepackage{array}
\usepackage{mdframed}

\newenvironment{conditions}
  {\par\vspace{\abovedisplayskip}\noindent\begin{tabular}{>{$}l<{$} @{${}={}$} l}}
  {\end{tabular}\par\vspace{\belowdisplayskip}}

\title{Fréquence de capitalisation et taux d'intérêt effectif : quel impact sur vos investissements ?}
\author{Gaëtan Bouget}

\begin{document}
\maketitle

Dans un environnement financier où les offres d’investissement se multiplient, comprendre le coût ou le rendement réel d’un produit est essentiel. Le taux d’intérêt effectif (AER) permet de comparer des supports avec des fréquences de capitalisation différentes, en intégrant l’effet des intérêts composés. Cet article propose une analyse comparative de deux stratégies d’investissement et étudie leur sensibilité aux fluctuations des taux.

\section{Mise en situation}
\subsection{Problème}
Je souhaite investir un capital initial $C_0$ mais j'hésite entre deux supports :
\begin{itemize}
\item Le support $S_a$ verse les intérêts annuellement.
\item Le support $S_q$  verse les intérêts quotidiennement.
\end{itemize}

\subsection{Objectif}
L'objectif de cet exercice est d'étudier le rendement des supports d'investissement et de modéliser les effets de différentes fréquences de capitalisation sur la valorisation du capital au fil du temps. Ce travail se divise en plusieurs étapes :

\begin{itemize}
    \item Étudier le rendement des supports $S_a$ et $S_q$ en fonction :
    \begin{itemize}
        \item du capital initial investi $C_0$ ;
        \item de la durée d'investissement exprimée en années $t$ ;
        \item du taux de rendement $r$.
    \end{itemize}
    \item Calculer le taux d'intérêt effectif de $S_q$ (Annual Equivalent Rate). Généraliser à toute période de capitalisation $n$, où $n$ est un entier positif représentant le nombre de capitalisations par an, avec $n \geq 1$ (capitalisation quotidienne pour $n = 365$ et continue lorsque $n \to \infty$).
\end{itemize}

\subsection{Hypothèses de travail}
\begin{itemize}
\item Les intérêts sont composés pour les deux supports.
\item Une année vaut 365 jours.
\item Le rendement annuel du support $S_a$ est noté $r_a$.
\item Le rendement quotidien du support $S_q$ est noté $r_q$.
\item Aucun frais ni taxe n'est appliqué.
\item Aucun versement ou retrait supplémentaire n’est effectué durant la période d'investissement (seul le capital initial $C_0$ est investi).
\end{itemize}

\section{Proposition d'une solution}
\subsection{Démarche méthodologique}
L'analyse se décompose en deux grandes parties :

\subsubsection*{Partie 1 : Le piège de l'équivalence naïve}
\begin{enumerate}
    \item Choisir \( r_q = \frac{r_a}{365} \).
    \item Calculer le capital final en utilisant la formule des intérêts simples, et constater que ce calcul mène à une apparente équivalence entre la capitalisation quotidienne et annuelle (ce qui est illusoire).
    \item Calculer le capital final avec la formule des intérêts composés pour la capitalisation quotidienne, et constater la différence par rapport aux intérêts simples.
    \item Déduire la différence entre les deux résultats, soulignant l'effet puissant des intérêts composés.
    \item Tracer des courbes représentant l'évolution du capital en fonction du temps afin de mettre en évidence cet écart.
    \item Énoncer la formule qui exprime la différence de rendement entre la capitalisation quotidienne et l'approche naïve.
    \item Démontrer que l'approche consistant à multiplier directement \( r_q \) par 365 est incompatible avec le fonctionnement réel de la capitalisation quotidienne. % preuve par l'absurde ou par développement binomial
\end{enumerate}

\subsubsection*{Partie 2 : Standardiser les comparaisons avec le taux annuel effectif (AER)}
\begin{itemize}
    \item Définir le taux annuel effectif (AER) qui intègre la fréquence de capitalisation. La formule générale est :
    \[
    AER = \left(1 + \frac{r_a}{n}\right)^n - 1,
    \]
    où \( n \) est le nombre de capitalisations par an (avec \( n = 365 \) pour la capitalisation quotidienne, et \( n \to \infty \) pour la capitalisation continue).
    \item Utiliser ce taux pour comparer équitablement les rendements des supports \( S_a \) et \( S_q \), ainsi que d'autres produits financiers, en offrant une mesure standardisée du rendement effectif.
\end{itemize}

\section{Pour aller plus loin}
Développer un programme qui simule l'évolution du capital en fonction du montant investi, du taux de rendement, de la fréquence de capitalisation des intérêts et de la durée d'investissement. Indiquer le taux d'intérêt effectif.

\end{document}
