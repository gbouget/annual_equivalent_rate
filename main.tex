\documentclass{article}

\usepackage[T1]{fontenc}
\usepackage[french]{babel}

\usepackage[letterpaper,top=2cm,bottom=2cm,left=3cm,right=3cm,marginparwidth=1.75cm]{geometry}

\usepackage{amsmath}
\usepackage{graphicx}
\usepackage[colorlinks=true, allcolors=blue]{hyperref}
\usepackage{eurosym}
\usepackage[group-separator={\ },output-decimal-marker={,}]{siunitx} 
\usepackage{amsfonts}
\usepackage{float}
\usepackage{pdflscape}
\usepackage{algorithm}
\usepackage{algorithmic}
\usepackage{array}
\usepackage{mdframed}

\newenvironment{conditions}
  {\par\vspace{\abovedisplayskip}\noindent\begin{tabular}{>{$}l<{$} @{${}={}$} l}}
  {\end{tabular}\par\vspace{\belowdisplayskip}}

\title{Impact de la fréquence de capitalisation sur le taux d'intérêt effectif et la valorisation des investissements}
\author{Gaëtan Bouget}

\begin{document}
\maketitle

\section{Problème}
Dans un contexte financier caractérisé par une diversification accrue des produits d’investissement, la variation des fréquences de capitalisation complique l’évaluation objective des rendements réels. Ainsi, deux actifs affichant un même taux nominal annuel peuvent générer des rendements distincts en raison des différentes fréquences de capitalisation, amplifiant ainsi l'effet des intérêts composés.

\section{Objectif}
L'objectif consiste à fournir une compréhension approfondie des mécanismes influençant le rendement réel des investissements et à établir une base solide pour la comparaison de diverses stratégies financières.

\section{Méthode}
L'approche adoptée pour atteindre l'objectif se décline en quatre étapes :
\begin{enumerate}
    \item mise en évidence des limites de la méthode de calcul des intérêts simples ;
    \item intégration des intérêts composés dans l'évaluation du rendement ; 
    \item démonstration, par développement binomial, de la supériorité des intérêts composés ;
    \item déduction de la formule du taux effectif annuel (TEA).
\end{enumerate}

Afin de faciliter les applications numériques avant de généraliser, l'analyse se concentre sur l'étude de deux scénarios distincts à partir d'un investissement initial \( C_0 = 100\,000\ \text{€} \) :
\begin{itemize}
    \item un actif financier \( A_a \) affichant un taux nominal annuel \( r_a = 5\% \) avec une capitalisation annuelle ;
    \item un actif financier \( A_q \) affichant un taux nominal annuel \( r_q = 5\% \) avec une capitalisation quotidienne (365 jours/an).
\end{itemize}

Enfin, chaque étape est déclinée en une série de questions destinées à approfondir la compréhension du phénomène.

\subsection{Hypothèses de travail}
Les conditions suivantes sont posées :
\begin{itemize}
    \item Une année est considérée comme comprenant 365 jours.
    \item Aucun frais ni taxe n'est appliqué.
    \item Aucun versement ou retrait supplémentaire n’est effectué durant la période d'investissement (seul le capital initial \( C_0 \) est investi).
    \item Le taux de rendement est constant dans le temps.
\end{itemize}

\section{Proposition d'une solution}
\subsection{Démarche méthodologique}
L'analyse se décompose en deux grandes parties :

\subsubsection*{Partie 1 : Le piège de l'équivalence naïve}
\begin{enumerate}
    \item Choisir \( r_q = \frac{r_a}{365} \).
    \item Calculer le capital final en utilisant la formule des intérêts simples, et constater que ce calcul mène à une apparente équivalence entre la capitalisation quotidienne et annuelle (ce qui est illusoire).
    \item Calculer le capital final avec la formule des intérêts composés pour la capitalisation quotidienne, et constater la différence par rapport aux intérêts simples.
    \item Déduire la différence entre les deux résultats, soulignant l'effet puissant des intérêts composés.
    \item Tracer des courbes représentant l'évolution du capital en fonction du temps afin de mettre en évidence cet écart.
    \item Énoncer la formule qui exprime la différence de rendement entre la capitalisation quotidienne et l'approche naïve.
    \item Démontrer que l'approche consistant à multiplier directement \( r_q \) par 365 est incompatible avec le fonctionnement réel de la capitalisation quotidienne. % preuve par l'absurde ou par développement binomial
\end{enumerate}

\subsubsection*{Partie 2 : Standardiser les comparaisons avec le taux annuel effectif (AER)}
\begin{itemize}
    \item Définir le taux annuel effectif (AER) qui intègre la fréquence de capitalisation. La formule générale est :
    \[
    AER = \left(1 + \frac{r_a}{n}\right)^n - 1,
    \]
    où \( n \) est le nombre de capitalisations par an (avec \( n = 365 \) pour la capitalisation quotidienne, et \( n \to \infty \) pour la capitalisation continue).
    \item Utiliser ce taux pour comparer équitablement les rendements des supports \( S_a \) et \( S_q \), ainsi que d'autres produits financiers, en offrant une mesure standardisée du rendement effectif.
\end{itemize}

\section{Pour aller plus loin}
Développer un programme qui simule l'évolution du capital en fonction du montant investi, du taux de rendement, de la fréquence de capitalisation des intérêts et de la durée d'investissement. Indiquer le taux d'intérêt effectif.

\end{document}
