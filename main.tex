\documentclass{article}

\usepackage[T1]{fontenc}
\usepackage[french]{babel}

\usepackage[letterpaper,top=2cm,bottom=2cm,left=3cm,right=3cm,marginparwidth=1.75cm]{geometry}

\usepackage{amsmath}
\usepackage{graphicx}
\usepackage[colorlinks=true, allcolors=blue]{hyperref}
\usepackage{eurosym}
\usepackage[group-separator={\ },output-decimal-marker={,}]{siunitx} 
\usepackage{amsfonts}
\usepackage{float}
\usepackage{pdflscape}
\usepackage{algorithm}
\usepackage{algorithmic}
\usepackage{array}
\usepackage{mdframed}

\newenvironment{conditions}
  {\par\vspace{\abovedisplayskip}\noindent\begin{tabular}{>{$}l<{$} @{${}={}$} l}}
  {\end{tabular}\par\vspace{\belowdisplayskip}}

\title{Fréquence de capitalisation et taux d'intérêt effectif : quel impact sur vos investissements ?}
\author{Gaëtan Bouget}

\begin{document}
\maketitle

Dans un environnement financier où les offres d’investissement se multiplient, comprendre le coût ou le rendement réel d’un produit est essentiel. Le taux d’intérêt effectif (AER) permet de comparer des supports avec des fréquences de capitalisation différentes, en intégrant l’effet des intérêts composés. Cet article propose une analyse comparative de deux stratégies d’investissement et étudie leur sensibilité aux fluctuations des taux.

\section{Mise en situation}
\subsection{Problème}
Je souhaite investir un capital initial $C_0$ mais j'hésite entre deux supports :
\begin{itemize}
\item Le support $S_a$ verse les intérêts annuellement.
\item Le support $S_q$  verse les intérêts quotidiennement.
\end{itemize}

\subsection{Objectif}
L'objectif de cet exercice est d'étudier le rendement des supports d'investissement et de modéliser les effets de différentes fréquences de capitalisation sur la valorisation du capital au fil du temps. Ce travail se divise en plusieurs étapes :

\begin{itemize}
    \item Étudier le rendement des supports $S_a$ et $S_q$ en fonction :
    \begin{itemize}
        \item du capital initial investi $C_0$ ;
        \item de la durée d'investissement exprimée en années $t$ ;
        \item du taux de rendement $r$.
    \end{itemize}
    \item Calculer le taux d'intérêt effectif de $S_q$ (Annual Equivalent Rate). Généraliser à toute période de capitalisation $n$, où $n$ est un entier positif représentant le nombre de capitalisations par an, avec $n \geq 1$ (capitalisation quotidienne pour $n = 365$ et continue lorsque $n \to \infty$).
    \item Développer un programme qui simule l'évolution du capital en fonction du montant investi, du taux de rendement, de la fréquence de capitalisation des intérêts et de la durée d'investissement. Indiquer le taux d'intérêt effectif.
\end{itemize}

\begin{center}
\rule{0.5\textwidth}{.4pt}
\end{center}

\textbf{Bonus}

Analyser l'impact de la volatilité du taux de rendement sur la valorisation du capital :

\begin{itemize}
    \item \textit{Modélisation des variations du taux de rendement :}  
    Supposer que le taux de rendement suit une distribution normale, avec une moyenne de 5\% et un écart-type de 1\%.  
    \item \textit{Impact des fluctuations annuelles :}  
    Étudier comment des variations de $\pm 1\%$ par an influencent le capital final après 10, 20 et 30 ans.  
    \item \textit{Simulation de Monte Carlo :}  
    Effectuer 10 000 simulations aléatoires pour analyser la sensibilité du capital final aux fluctuations du taux de rendement pour les supports $S_a$ et $S_q$.  
    \item \textit{Mesure des risques :}  
    \begin{itemize}
        \item Calculer la Value at Risk (VaR) après 30 ans avec un niveau de confiance de 95\%.  
        \item Déterminer les intervalles de confiance à 95\% pour le capital final et estimer les bornes inférieure et supérieure après chaque période d’investissement.  
        \item Mesurer l’écart-type des rendements annuels pour évaluer la volatilité du capital dans un environnement incertain.  
    \end{itemize}
    \item \textit{Visualisation des résultats :}  
    \begin{itemize}
        \item Générer des courbes de distribution du capital final pour $S_a$ et $S_q$ sur 10 000 simulations.  
        \item Comparer la dispersion des rendements à l’aide d’histogrammes et de boxplots afin d’évaluer la stabilité relative des investissements.  
    \end{itemize}
\end{itemize}

\subsection{Hypothèses de travail}
\begin{itemize}
\item Les intérêts sont composés pour les deux supports (réinvestis immédiatement après leur versement).
\item Une année vaut 365 jours (pas de prise en compte des années bissextiles).
\item Le rendement annuel du support $S_a$  est noté $r_a$.
\item Le rendement quotidien $r_q$ de $S_q$ vaut $\frac{r_a}{365}$.
\item Aucun frais ou taxe n'est appliqué.
\item Aucun versement ou retrait supplémentaire n’est effectué pendant la durée de l’investissement (seul le capital initial $C_0$ est placé).
\end{itemize}

\end{document}
