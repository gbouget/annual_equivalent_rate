\documentclass{article}

\usepackage[T1]{fontenc}
\usepackage[french]{babel}

\usepackage[letterpaper,top=2cm,bottom=2cm,left=3cm,right=3cm,marginparwidth=1.75cm]{geometry}

\usepackage{amsmath}
\usepackage{graphicx}
\usepackage[colorlinks=true, allcolors=blue]{hyperref}
\usepackage{eurosym}
\usepackage[group-separator={\ },output-decimal-marker={,}]{siunitx} 
\usepackage{amsfonts}
\usepackage{float}
\usepackage{pdflscape}
\usepackage{algorithm}
\usepackage{algorithmic}
\usepackage{array}
\usepackage{mdframed}

\newenvironment{conditions}
  {\par\vspace{\abovedisplayskip}\noindent\begin{tabular}{>{$}l<{$} @{${}={}$} l}}
  {\end{tabular}\par\vspace{\belowdisplayskip}}

\title{Fréquence de capitalisation et taux d'intérêt effectif : quel impact sur vos investissements ?}
\author{Gaëtan Bouget}

\begin{document}
\maketitle

Dans un environnement financier où les offres d’investissement se multiplient, comprendre le coût ou le rendement réel d’un produit est essentiel. Le taux d’intérêt effectif (AER) permet de comparer des supports avec des fréquences de capitalisation différentes, en intégrant l’effet des intérêts composés. Cet article propose une analyse comparative de deux stratégies d’investissement et étudie leur sensibilité aux fluctuations des taux.

\section{Mise en situation}
\subsection{Problème}
Je souhaite investir un capital initial $C_0$ mais j'hésite entre deux supports :
\begin{itemize}
\item Le support $S_a$ verse les intérêts annuellement.
\item Le support $S_q$  verse les intérêts quotidiennement.
\end{itemize}

\subsection{Objectif}
L'objectif de cet exercice est d'étudier le rendement des supports d'investissement et de modéliser les effets de différentes fréquences de capitalisation sur la valorisation du capital au fil du temps. Ce travail se divise en plusieurs étapes :

\begin{itemize}
    \item Étudier le rendement des supports $S_a$ et $S_q$ en fonction :
    \begin{itemize}
        \item du capital initial investi $C_0$ ;
        \item de la durée d'investissement exprimée en années $t$ ;
        \item du taux de rendement $r$.
    \end{itemize}
    \item Calculer le taux d'intérêt effectif de $S_q$ (Annual Equivalent Rate). Généraliser à toute période de capitalisation $n$, où $n$ est un entier positif représentant le nombre de capitalisations par an, avec $n \geq 1$ (capitalisation quotidienne pour $n = 365$ et continue lorsque $n \to \infty$).
    \item Développer un programme qui simule l'évolution du capital en fonction du montant investi, du taux de rendement, de la fréquence de capitalisation des intérêts et de la durée d'investissement. Indiquer le taux d'intérêt effectif.
\end{itemize}

\begin{center}
\rule{0.5\textwidth}{.4pt}
\end{center}

\textbf{Bonus :} étudier la sensibilité du taux de rendement.
\begin{itemize}
    \item Utiliser une distribution normale pour modéliser les variations aléatoires du taux de rendement, avec un écart-type de 1\% autour du taux moyen de 5\%.
    \item Évaluer l'impact de petites variations du taux de rendement annuel (par exemple, +1\% ou -1\% chaque année) sur le capital final après 10, 20 et 30 ans.
    \item Analyser la sensibilité du capital final à des variations aléatoires du taux de rendement en utilisant une simulation de Monte Carlo avec 10 000 itérations pour les supports $S_a$ et $S_q$.
    \item Calculer la Value at Risk (VaR) du capital après 30 ans pour les deux supports d'investissement (annuel et quotidien), avec un niveau de confiance de 95\%.
    \item Construire des intervalles de confiance à 95\% pour le capital final des deux supports, en estimant les bornes inférieure et supérieure pour chaque période de simulation.
    \item Mesurer l'écart-type des rendements annuels pour chaque support afin d'analyser la volatilité du rendement dans un environnement volatile.
    \item Visualiser les résultats sous forme de graphiques :
    \begin{itemize}
        \item Générer des courbes de distribution du capital final pour $S_a$ et $S_q$ sous 10 000 simulations aléatoires.
        \item Comparer les distributions des rendements des deux supports en utilisant des histogrammes et des boxplots pour évaluer la stabilité des investissements.
    \end{itemize}
\end{itemize}

\end{document}
